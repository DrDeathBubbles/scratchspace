\documentclass[10pt,a4paper]{article}
\usepackage[utf8]{inputenc}
\usepackage[english]{babel}
\usepackage{amsmath}
\usepackage{amsfonts}
\usepackage{amssymb}
\usepackage{graphicx}
\usepackage{float}

\usepackage{amsmath}
\usepackage{amsfonts}
\usepackage{amssymb}
\usepackage{graphicx}
\usepackage{siunitx}
\usepackage[affil-it]{authblk}


\title{Temporal evolution of a monodisperse foam stabilised against coarsening.}

\author[a]{A. J. Meagher\footnote{Email:meagheaj@tcd.ie; Fax:004431428611}}
\author[d]{D. Whyte}
\author[a,b]{J. Banhart}
\author[d]{S. Hutzler}
\author[d]{D. Weaire}
\author[a,b]{F. Garc\'{i}a-Moreno}
\affil[a]{Technische Universit{\"a}t Berlin, Hardenbergstrasse 36, 10623 Berlin, Germany}
\affil[b]{Institute of Applied Materials, Helmholtz Centre Berlin for Materials and Energy, Hahn-Meitner-Platz 1, 14109 Berlin, Germany.}
\affil[c]{Institute of Mathematics and Physics, Aberystwyth University, Penglais, Aberystwyth,
Ceredigion, Wales, SY23 3BZ, United Kingdom}
\affil[d]{School of Physics, Trinity College Dublin, Dublin 2, Ireland}
\renewcommand\Authands{ and }


\begin{document}

\maketitle


\begin{abstract}

The evolution of a three-dimensional monodisperse foam was investigated using X-ray tomography over the course of seven days. The coarsening of the sample was inhibited through the use of perfluorohexane gas. The internal configuration of bubbles is seen to change markedly, evolving from a disordered arrangement towards a more ordered state. We chart this ordering process through the use of the coordination number, the Bond Orientational Order Parameter (BOOP) and the translational order parameter. 

\end{abstract}

\section{List of keywords}

\begin{itemize}

\item Monodisperse foam
\item Crystallisation
\item BOOP
\item Bragg

\end{itemize}

\section{Introduction}

The ordering behaviour of bulk samples of equal-sized bubbles less than 1 mm in diameter -- known as monodisperse microfoams -- 
was first discovered somewhat incidentally by Bragg and Nye in the 1940s \cite{Bragg47}, in the context of their work on the two-dimensional  bubble raft. The latter has remained popular up to the present day \cite{Tolen86,Gouldstone2001,doi:10.1080/01418610208235711}, but only recently has the nature of three-dimensional bubble crystals begun to be explored. Many key questions about their nature remain unanswered. 
Optical experiments \cite{HohlerEtal08,VanderNet06,vanderNetEtal07} have given only a superficial indication  of structure, enough to stimulate theories that seek to explain an apparent preference for the fcc structure \cite{Heitkam2012}. 
The development of advanced 3D imaging techniques has been the key to structural investigation of the internal arrangements in aqueous foams. In particular, X-ray tomography has been employed with much success \cite{lambert2007experimental,Stocco_etal2011}. While such experiments have previously been confined to synchrotron facilities, where limited experimental time can severely restrict the scope of experiments, we have shown that technological advancements now allow bench-top X-ray tomography to image wet aqueous foams \cite{Meagher2011}. Initial experiments showed that the internal structure of the sample is more complicated than that of the surface layers. 
In this paper, we expand upon this earlier work by investigating the ordering behaviour of a monodisperse microfoam composed of roughly 11,000 bubbles which we image successively over the course of seven days. Through the use of perfluorohexane, the coarsening rate of the sample was lowered sufficiently for the sample to be considered monodisperse over the course of the experiment. In addition to modifying the structure, coarsening would give rise to changes during the image acquisition phase (2 hour duration), leading to significant blurring in the final 3D images. 
In analysing the data, we use various measures of average and local structure, including the coordination number, the bond orientational order parameter and the translational order parameter. These metrics allow us to precisely characterise and chart the evolving structure of our foam sample. In this way, we hope to demonstrate the use of 3D monodisperse microfoams as a model system for the examination and demonstration of crystal structures and their evolution in general, just as 2D rafts have been employed \cite{Bragg47}, ever since Bragg introduced them for that purpose. 
At odds with our expectation, the experiments revealed that the structure of the sample was not static over the experimental lifetime. Instead, the structure in the centre of the sample was seen to evolve from a disordered state on the first day of the experiment to a relatively ordered state on the seventh day. This remains surprising, since the mechanism of recrystallisation is not obvious.



\section{Experimental Method}

Monodisperse bubbles were produced using a flow-focusing device \cite{Ganan2004update,VanderNet06,Smith49}. The device, based on the co-flow of surfactant solution and pressurised gas, can produce monodisperse bubbles of diameter between \SI{20}{\um} and 2 \SI{2}{\mm} by varying the liquid flow rate, the gas pressure and outlet-nozzle diameter. For the purposes of experiment, a sample may be considered `monodisperse' if the dispersity of the sample is less than 5\% \cite{HohlerEtal08}.  Our surfactant solution was composed of a 5\% by volume commercially-available detergent \emph{Fairy Liquid} in water. This has been previously found to produce stable foams suitable for a wide variety of foam experiments. Our gas phase was formed of oxygen-free nitrogen into which the relatively insoluble compound perfluorohexane was dissolved to reduce the  coarsening rate of the foam \cite{WeairePageron90}.  

The flow-focusing device was attached to the bottom of a large rectangular box which was filled with the surfactant solution. The device was adjusted to produce monodisperse bubbles, with a diameter less than 1 mm. A cubic container, with one open face, of internal dimensions \SI{20}{\mm} x \SI{20}{\mm} x \SI{20}{\mm} was placed into our surfactant solution. The container was then inverted to remove trapped air, before being positioned, open-face down, over the outlet of the device. This allowed for the bubbles produced by our flow-focusing device to be collected without exposing them to atmosphere, preventing their rapid expansion \cite{FortesDeus95}. Once filled so as to produce a foam sample approximately 12 bubble-layers deep, the container was sealed by sliding a perspex plate over the open face. The container was removed from the solution, dried, and glued to a plastic plinth which was then affixed to the rotation stage of our $\mu$CT tomographic imaging device. The sample was allowed to settle for two hours after production before being imaged. Previous experiments have shown that several rearrangements occur during this settling time which would cause blurring during tomographic imaging. Note that this method of preparation is rather different from some of our earlier work, in which foam was rapidly ejected onto the surface of a pool of solution, and no solid boundaries were involved.

%The flow-focusing device was attached to the bottom of a container which was filled with our soap-solution. This allowed for the bubbles produced by our flow-focusing device to be collected without exposing them to atmosphere, preventing their rapid-expansion \cite{FortesDeus95}. The flow-focusing device was adjusted to produce bubbles, monodisperse, with an average diameter less than 1 $mm$. A cubic container of internal dimensions 20 $mm$ x 20 $mm$ x 20 $mm$ was placed into our soap solution and inverted to remove trapped air. The container was then placed, open-face down, over the flow-focusing outlet. Once filled so as to produce a foam sample approximately 15 bubble-layers deep, the container was sealed by sliding a perspex plate over the open face. The container was removed from the soap solution, dried, and glued to a plastic plinth which was then inserted into our $\mu CT$ tomographic imaging device. The sample was allowed to settle for two hours after production before being imaged. Previous experiments have shown that several re-arrangements occur during this settling time which would cause blurring during tomographic imaging.

Each tomography took approximately two hours to complete, after which the sample was removed from the chamber before being imaged the next day, resulting in a 22 hour period between each tomographic image. The sample was removed from the device during this time to allow other experiments to be conducted. 


Our imaging device was composed of a micro-focus 150 kV Hamamatsu X-ray source with a tungsten target. The sample was mounted on a precision rotation stage from Huber Germany, the sample's radioscopic projections recorded using a flat panel detector C7942 (\SI{120}{\mm} x \SI{120}{\mm}, 2240 x 2368 pixels, pixel size 50 $\mu$m). Different magnifications of the sample are possible by adjusting the relative distances between the X-ray source, the sample and the detector.
By varying the filament voltage and current, a 100 kV filament voltage and a \SI{100}{\uA} were found to provide the best contrast and lowest noise in the reconstructed foam images at high spatial resolution.

\begin{figure}[ht!]
\centering
\includegraphics[width=0.5\textwidth]{../Structure/Temporal/Temporal_visualisation_5}
\caption{Visualisation of the gas-phase of the sample showing the bottom of the sample in contact with the foam-liquid interface. Gravity acts perpendicular to this surface in the z direction. Around the boundary of the foam sample areas of hexagonal closed-packed ordering are seen to occur, while the central region appears to be disordered in nature.}
\label{fig:Sample_visualisation}
\end{figure}

Before taking the 3$^{rd}$ tomography, the sample was accidentally disturbed while being mounted. Only after the experiment had finished was the extent of disturbance apparent. However, due to the startling and unobserved nature of the behaviour seen in the analysis, the original experimental data has been used for this publication. 

This absorption profile may be directly related to the liquid content of the sample via the Beer–Lambert law \cite{Garcia2013}. 
X-ray tomographic reconstruction was performed using the commercially available software \emph{Octopus} \cite{Octopus_reference}. The image slices were further processed using the image processing software \emph{MAVI}, allowing sample characteristics such as bubble volume, position etc. to be extracted \cite{MAVI}. By approximating the bubble volumes as spheres we can compute the bubble radius. The average bubble diameter and standard deviation was then determined by fitting a Gaussian curve to the resulting bubble diameter distribution. The sample was visualised using VG studio max \cite{website:VGstudiomax}. A reconstruction of the gas phase of the foam is shown in Fig. \ref{fig:Sample_visualisation}.

After performing our image analysis, we disregarded those objects whose diameter was outside one standard deviation of the mean associated with the bubbles of the experiment. This criterion was used to identify the bubbles of our foam for further analysis, while allowing us to ignore most noise associated with the image segmentation process.

\section{Results}
\label{sec:results}

The average bubble diameter increases slightly over the duration of the experiment, rising from 794 $\mu$m to 815 $\mu m$. During this time, the dispersity does not rise above 5\%, thus classifying the foam as effectively monodisperse \cite{HohlerEtal08}.

The liquid fraction $\Phi_l$ of the sample was monitored by investigating the X-ray absorption profile of the sample. This absorption profile may be directly related to the liquid content of the sample via the Beer-Lambert law \cite{Garcia2013}. Our analysis shows that the average liquid fraction of the sample decreases slightly from its initial value of 0.2 over the course of the experiment. 
 



%When preforming our image analysis we disregarded those objects whose size was outside one standard deviation of the mean associated with the bubbles of the experiment. This criterion was used to select the bubbles of our foam for further analysis, while allowing us to ignore most noise associated with the image segmentation process.

While the average diameter of the bubbles of the sample did not change significantly, the internal structure underwent significant alterations. Plots of the bubble centre positions are shown in Fig.\ref{fig:xy_centers}. In such plots, the z coordinates of the bubbles are ignored, resulting in the overlay of all foam layers. 

On the first day of the experiment (Fig.\ref{fig:xy_centers} (1)), the bubbles are seen to arrange into two distinct regions: near the container walls, linear arrangements of bubbles are seen while no such collection of points is seen in the sample centre. This indicates that the outer layers of the foam sample are ordered, while the inner region is disordered in nature, as keeping with our previous experiments \cite{Meagher2011}.
Incoherent grain boundaries are seen to form, separating the four ordered regions at the walls of the sample.

 
Over the lifetime of the experiment, the ordered arrangements of bubbles are seen to increase in extent (Fig.\ref{fig:xy_centers} day 2 to day 7), encroaching on the disordered centre of the sample. 

\begin{figure}[H]
\centering
\includegraphics[width=0.7\textwidth]{../Structure/Temporal/temporal_3f_low}
\caption{Plot of all the centre positions of the bubbles, projected on a horizontal plane, over seven days. On the first day it is seen that the projected positions are arranged in lines parallel to the sides of the container while the centre of the sample appears to be disordered. As the experiment progresses, the outer ordering of the sample is seen to encroach upon the central region.}
\label{fig:xy_centers}
\end{figure}

To characterise the various structures and transitions which were occurring within the sample over the seven days of the experiment, several order metrics were calculated based on the bubble centre positions. In particular, we investigated the \emph{Coordination Number} $n$, the \emph{Translational Order Parameter} and the \emph{Bond Orientational Order Parameter} (BOOP), as described below.


\subsection{Coordination number}

The local coordination number $n$ is the number of nearest neighbours for each particle. There exist several definitions of `nearest neighbour' e.g. those objects within a packing which share a face of the corresponding Voronoi diagram \cite{Olafsen2010}, or those objects at a distance corresponding to the first minima of the radial distribution function \cite{Aste2005}. For ease of interpretation, however, we consider two bubbles as neighbours if

\begin{equation}
|\vec{r}_i - \vec{r}_j| \le R_i + R_j
\label{eqn:Neighbourhood}
\end{equation}

where $\vec{r}_i$ and $\vec{r}_j$ are the positions of the $i^{th}$ and $j^{th}$ bubbles within the system and $R_i$ and $R_j$ their respective radii. 
 
\begin{figure}[H]
\centering
\includegraphics[width=0.80\textwidth]{../Structure/Temporal/Coord_final}
\caption{Variation of the coordination number distribution over the lifetime of the experiment. The distributions have a maximum around $n=12$, indicating closed-packed ordering. On the first and third day of the experiments, the wider distributions when compared to those of the other days.}

%\caption{Variation of the coordination number distribution over the lifetime of the experiment. When comparing the data, we see wide distributions on the first and third indicating a disordered structure. This wide distribution evolves towards a more peaked distribution on the second and seventh day, indicating the ordering of the sample.}

\label{fig:coord_distribution}
\end{figure}

We calculate the coordination number for roughly 5000 bubbles within a cubic region at the centre of the sample, hence avoiding boundary effects. The probability distribution $P(n)$ of coordination number $n$ over the seven days of the experiment is shown in Fig. \ref{fig:coord_distribution}.
As the experiment progresses, the first broad peak, seen on day 1, evolves towards a more narrow distribution with a peak around $n$ =12. The peak widens again on the third day, following the samples disturbance before narrowing again over the coming days.

\subsection{Translational Order Parameter}

The translational order parameter $G$ is a measure of the spatial symmetry of a packing, based on the ratio of the first minimum and maximum of the radial distribution function $g(r)$ \cite{Olafsen2010}. For the case of a perfectly ordered sample, $g(r)$ will be formed from a sum of $\delta$ functions. As the level of disorder increases, these $\delta$ functions increase in width, leading to a continuous distribution. For such a distribution,  
$G$ may be defined as

\begin{equation}
G = \left|\frac{g(r_{g1})}{g(r_{g2})} \right|
\label{eq:translational}
\end{equation}

where $r_{g1}$ and $r_{g2}$ are the positions of the first minimum and maximum of the RDF respectively. For a perfectly crystalline sample, $G=0$. $G$ increases in value as the level of translational disorder within the system increases.

Fig.\ref{fig:translational_order_parameter} shows a plot of the translational order parameter as calculated over the seven days of the experiment. The values of $g(r_1)$ and $g(r_2)$ were calculated using 4$^{th}$ order polynomial fits to the radial distribution functions calculated from the experimental data. 

It is seen that $G$ slightly decreases over the first 2 days of the experiment, before rising dramatically on the third day, corresponding to the shaking of the system. Following this, $G$ is seen to strongly decrease for the remainder of the experiment. 

\begin{figure}[H]
\centering
\includegraphics[width=0.80\textwidth]{../Structure/Temporal/temporal_4b.png}
\caption{Variation of the translational order parameter, $G$ Eq.\ref{eq:translational} over the seven days of the experiment. The translational order parameter is seen to decrease over the fist two days of the experiment, before rising on the third day of the experiment due to a mechanical disturbance of the sample. Following this, $G$ appears to decrease smoothly over the next five days of the experiment.}
\label{fig:translational_order_parameter}
\end{figure}


\subsection{Analysis using Bond Orientational Order Parameter}

The BOOP or Steinhardt order parameter is a measure of the local rotational
order within a sample \cite{PhysRevB.28.784}. Although there exist several methods by which this
rotational symmetry may be classified, it is found that Steinhardt's characterisation has proven the most useful in a variety of simulations and experiments of granular systems \cite{lechner:114707,Aste2005}. However, this type of analysis has not yet been applied to foams due to a lack of three-dimensional data of sufficiently high resolution.

The order parameter $Q_\ell$ for a bubble $a$ is defined by:
\[ Q_\ell(a) = \sqrt{\frac{4\pi}{2\ell+1}\sum_{m=-\ell}^\ell \left| \frac{1}{n(a)} \sum_{b \in \text{NN}(a)} Y_{\ell m}(\theta_{ab},\phi_{ab}) \right|^2}, \]

where $n(a)$ is the number of nearest neighbours, $\text{NN}(a)$, of the bubble $a$, $\phi_{ab}$ and $\theta_{ab}$ are the polar and azimuthal angles associated with the vector from $a$ to its neighbour $b$, $Y_{\ell m}$ is the  spherical harmonic. The cutoff radius for classification of nearest neighbours
is obtained from the first minimum of the radial distribution function $g(r)$. Of particular relevance for us are the cases $\ell=4$ and $\ell=6$, which probe for cubic and hexagonal symmetry respectively.

For the hcp and fcc structures respectively we can calculate $(Q_4, Q_6)$ values analytically, as hcp: $(0.097,0.485)$, and fcc: $(0.191,0.574)$.

Some shortcomings of the BOOP method have recently been identified by Mickel \emph{et. al.} due to the strong dependence on the choice of nearest neighbours \cite{maestro2013liquid}. While we acknowledge the advantages of their proposed Minkowski structure method, we find that using the BOOP method is
sufficient to characterise our samples: see later discussion.


Fig.\ref{fig:Boop_comparison} shows the distribution of $(Q_4,Q_6)$ values for our sample as computed on days 1, 4 and 7. On day 1 we see a wide distribution of values, by day 4 two peaks are visible, which become sharper by day 7. The positions of the peaks --- at $(0.21,0.58)$ and $(0.14,0.50)$ --- are close to the theoretical values for ABC (fcc) and ABA (hcp) structures. Visual inspection suggests that fcc is dominant; we will return to this later.

\begin{figure}[H]
\centering
\includegraphics[width=0.7\textwidth]{./New_BOOP_figures/New_Q46_day_log_0}
(a)
\includegraphics[width=0.7\textwidth]{./New_BOOP_figures/New_Q46_day_log_3}
(b)
\includegraphics[width=0.7\textwidth]{./New_BOOP_figures/New_Q46_day_log_6}
(c)
\caption{3D plots showing the distribution of values of the $Q_4$ and the $Q_6$ parameters on the 1$^{st}$ (a), 4$^{rd}$ (b) and 7$^{th}$ (c) days of the
experiment. The wide distribution of $Q_4$ and $Q_6$ seen on the first day,(a), begins to show a two-peaked distribution (b). These two-peaks are centered around the $Q_4$ $Q_6$ values associated with FCC and HCP arrangements. By the seventh day of the experiment (c), the peaked distribution has continued to develop a clear preference for the formation of the FCC structure indicated by the relative increase in the height of this peak with respect to that of
the hcp lattice.}
%\caption{2D histograms of the $Q_4$ and $Q_6$ BOOP parameter as measured on the 3$^{\text{st}}$ and 7$^{\text{th}}$ days of the experiment. The data is presented using a log color scale. The wide distribution of values seen on the third day of the experiment migrates towards a peaked distribution on the seventh day.}
\label{fig:Boop_comparison}
\end{figure}

Fig. \ref{fig:Boop_coloring_1} shows a section excised from near the middle of the sample. Each bubble (displayed as a sphere) is coloured according to its $(Q_4, Q_6)$ values, based on a threshold distance in the $Q_4$-$Q_6$ plane. We see the emergence of regions of fcc and hcp by day 7. Visually, we see that this classification is correct.

\begin{figure}[H]
\centering
\includegraphics[width=1.0\textwidth]{./New_boop_coloring}
\caption{Bubbles excised from near the centre of the sample on days 1, 4 and 7 respectively. These bubbles are coloured according to their $(Q_{4},Q_{6})$ values: red for fcc, blue for hcp, white for other.}
\label{fig:Boop_coloring_1}
\end{figure}

Using the same methodology, we can plot projections of the positions of all the bubbles in the sample in order to show where ordering occurs within the sample: see Fig.\ref{fig:Boop_coloring_2}.

\begin{figure}[H]
\centering
\includegraphics[width=0.8\textwidth]{./xyz/Test2}
(a)
\includegraphics[width=0.8\textwidth]{./xyz/Test6}
(b)
\caption{Plot showing the distribution of fcc (red +) and hcp (black x) ordered regions within the foam sample on the third and seventh day of the experiment. Views of the $xy$, $yz$ and $xz$ planes are displayed. The extent of the crystallisation areas is seen to increase, but no clear distinction between
areas of fcc and hcp ordering occurs. }
\label{fig:Boop_coloring_2}
\end{figure}

Both figures show that regions of fcc and hcp ordering coexist in the sample. As the experiment progresses, the extent of these ordered regions increases, with a clear preference for fcc ordering over hcp. This is more clearly demonstrated by plotting the fraction of bubbles classified as either fcc or hcp
over time, as in Fig.\ref{fig:Banhart_graph}. While the initial ratio of fcc to hcp ($\sim 1.3$) is in line with previous experiments \cite{VanderNet06,vanderNetEtal07}, it increases dramatically with time, up to $\sim 2.4$.

\begin{figure}[H]
\centering
\includegraphics[width=1.0\textwidth]{./Banhart_graph}
\caption{Graph showing the variation of the ratios $\frac{N_{fcc}}{N_{total}}$ and $\frac{N_{hcp}}{N_{total}}$. The fraction of hcp ordering remains fouchly constant while the fraction of fcc ordering rises.}
\label{fig:Banhart_graph}
\end{figure}

\section{Discussion}

The average bubble diameter increases by 3\% over the 7 days of the experiment due to coarsening \cite{stevenson2012foam}. However, the dispersity of the sample never rises above 5\%, the experimental limit of a monodisperse foam \cite{HohlerEtal08}. Previous experiments on 3d foams formed without the addition of a low-solubility gas phase, have shown significantly higher coarsening rate \cite{Gonatas1995}. We can conclude that the PFH has indeed reduced the coarsening of the foam.

Each order parameter calculated indicates the ongoing ordering process occurring within the system. 

This is first shown by the hexagonal patterns present within the xy plots of center positions (Fig.\ref{fig:xy_centers} ). The regular arrangement of points in proximity to the border of the sample indicate crystalline structures in these areas, while the lack of such arrangements within the center of the sample indicate this region is disordered. The exact nature of this crystallisation is determined by BOOP analysis. This clearly shows that the crystalline regions are composed of fcc and hcp regions. Initially, there appears to be equal occurrence of fcc and hcp regions within the sample at the beginning of the experiment. By the 7th day of the experiment, a clear preference for fcc ordering is indicated by the increased ratio of the BOOP associated with this ordering compared to that associated with a hcp structure. This is in keeping with theoretical discourse about the preference for fcc structures over hcp ordering due to mechanical stability \cite{Heitkam2012}. 

It must be noted, however, that the BOOP signature of fcc and hcp ordering within our sample is shifted slightly with respect to the values associated with these structures in an ideal crystalline lattice. We suscpected that this shift was due to the finite compressibility of our bubbles. To invesstigate the validity of this assumption, we calculated the BOOP signature associated with a deformed fcc structure. Namely, we calculated the BOOP signature of an fcc lattice for which the lattice spacing in the $< 100 >$ has been successively reduced. 
We found that, as the compression of the sample increased, the corresponding BOOP values spread in a similar way as those found in experiment. xy plots of the regions of fcc and hcp ordering show no clear preference for fcc ordering in one area with respect to another. 

The coordination number of the sample shows that ordering process as the wide distributions on the 1st and 3rd days narrow. The probability of bubbles having 13 neighboursor more is in keeping with previous experiments on deformable spheres with packing fractions above 0.74 \cite{PhysRevE.84.011302}.

The translational order parameter, $G$, may be used to examine the rate of crystallisation within the sample. Over the last 5 days of the experiment, the decreasing rate at which $G$ changes indicates that the rate of structural change is also decreasing. This is to be expected as the region of disorder decreases.

The driving force behind this crystallisation is still undetermined. While coarsening dynamics have been previously linked to relaxation dynamics, the significantly reduced coarsening rate present within our foam suggests this is an unlikely source of the structural re-arrangements reported here. In addition, the increased ordering rate of the foam following physical disturbances of the sample is incongruent with such coarsening arguments. It is believed that thermal fluctuations in the lab over a 24 hour period could result in the behaviour observed. The warming and cooling of the lab during the day and night leads to successive cycles of expansion and contraction of the bubbles of the sample. This could produce the stimulation required to partially crystallise the sample.

The drainage of the sample may also be implicit in this ordering process. Liquid drainage has been previously linked to local rearrangements of  bubbles \cite{Carrier2003,PhysRevLett.91.188303}. In addition, as the shear modulus of a foam is inversely proportional to liquid fraction, drainage results in the occurrence of rearrangements in drier foams to be more difficult. The decrease in liquid fraction of our sample over time would result in a reduction in the crystallisation rate, as we see from the translational order parameter. 


%The packing fraction of the sample shows some interesting characteristics. Firstly the packing fraction does not vary significantly over the seven days of the experiment, regardless of the ordering characteristics of the foam. In addition, the packing fraction resides above the value $\phi = 0.74$, the maximum possible packing fraction of rigid spheres \cite{HalesHHMNOZ-DCG}. The reason for this high packing fraction may be determined by looking at the local packing fraction. Such an analysis, shown in Fig.\ref{fig:Voronoi_8}, indicates that the packing fraction increases with height within the sample. For a given layer within the foam, the buoyancy force of the underlying foam forces bubbles of upper layers closer together. This has the effect of increasing the packing fraction of these higher foam layers, as seen in experiment. Such an increase in packing fraction due to an applied force has been previously seen in experiments of deformable spheres \cite{PhysRevE.84.011302}. 

%This deformation of the bubbles is again demonstrated in the coordination number of the sample. While the narrowing of the distributions over the experimental lifetime indicates the ongoing ordering of the sample, it is also seen that there exists a significant probability that particles with $n=13$ neighbours exist within the packing. This high coordination is unphysical for hard spheres, indicating that significant deformation of our bubbles is occurring.


%The translational order parameter $G$ allows the speed at which the foam's structure is changing to be examined. Fig.\ref{fig:translational_order_parameter} shows that after the sample has been disturbed, the level of disorder within the sample increases, as shown by the comparatively hight $G$ on the third day of the experiment. Following this, $G$ is seen to decrease, showing the ongoing ordering of the sample. Note, however, that the rate of decrease of $G$ slows over the lifetime of the experiment, indicating a reduction in the bubble re-arrangement rate within the sample.  

%The translational order parameter $G$ allows the speed at which the foam's structure is changing to be examined.  Fig.\ref{fig:translational_order_parameter} shows two distinct periods of ordering within the sample - before and after the sample has been shaken. After the sample has been shaken, the sample is seen to increase in disorder, corresponding to an increase in the translational order parameter. Over the next five days of the experiment, the ordering of the sample increases leading to a corresponding decrease in the translational order parameter. During this time, however, the rate at which the $G$ decreases is seen to decrease, indicating a reduction in the ordering rate of the sample.

%The local and specific ordering of the sample, not accessible through the use of the coordination number or translational order parameter is provided by the Bond Orientational analysis. The regions of fcc and hcp ordering may be clearly identified and separated from the disordered bubbles. It is noted that when the BOOP signature of the sample is calculated, two peaks exists near the positions associated with fcc and hcp ordering. The experimental values are seen to spread away from those values associated with an ideal fcc and hcp lattice. To determine the origin for this spread in values, we calculated the BOOP order parameters associated with a deformed fcc lattice. By successively reducing the lattice spacing along the $<100>$ direction, we produce similar structures which occur due to the progressive compression of the sample. We found that, as the compression of the sample increased, the corresponding BOOP values spread in a similar way as those found in experiment.     

%We see that regions of fcc and hcp coexist within the sample and that there is no clear distinction between them. The corners of the sample are seen to remain disordered throughout the experiment's lifetime. This disordering may be in fact the existence of foam grain boundaries. The flat sides of the fcc lattice form the ideal template for the growth of an fcc lattice in the $<100>$ direction. A miss-match at the corners between the grown of these independent fcc lattices would result in these disordered regions. Also, a region of disorder is seen to separate the four boundary regions of crystallisation from the central ordered region on the seventh day of the experiment, indicating to us that this central region may be the product of a separate nucleation sight.

%All of our various order metrics show that our sample evolves towards a more ordered state with time. The ordering of similar systems of particles, such as colloidal suspensions, has been previously noted \cite{C3SM50500F,VanBlaaderen1997}, however the thermodynamic arguments put forward for their ordering may not be applied to our system. Due to the relatively large diameter of our bubbles, and the high surface tension of the liquid phase used, our foams are considered athermal \cite{PhysRevLett.91.188303}. Brownian motion cannot be seen as a driving force for the structural changes found. Other mechanisms of bubble re-arrangements must be investigated.

%Coarsening has been previously shown to cause significant structural changes within foams, the changing size of bubbles leading to localised increases in stress which is released during bubble re-arrangements \cite{citeulike:3169874}. However, the rate of these re-arrangements due to coarsening decreases in time \cite{PhysRevLett.91.188303}. Our data shows that the ordering rate of the foam sample does indeed decrease, however the shaking of the sample provokes a significant increase in the ordering rate which is inconsistent with this argument. Additionally, the PFH gas has significantly reduced the coarsening rate of our foam when compared to studies of other 3D foams coarsening. Our low coarsening rate, we believe, would not be sufficient to provoke the manifold re-arrangements required to order the sample. 

%We seen, however, that the liquid fraction of the foam changes over the lifetime of the experiment. Drainage of liquid through a foam has also been linked to local re-arrangements of bubbles \cite{Carrier2003}. As the shear modulus of a foam is inversely proportional to liquid fraction, increased liquid fraction decreases the shear modulus, allowing the bubbles to move more easily under buoyancy forces \cite{PhysRevLett.91.188303}. Our liquid fraction analysis shows that the sample decreases in liquid fraction content by a factor of 2 over the lifetime of the experiment. During this time, we see a corresponding decrease in the ordering rate as indicated by the retardation of the translational order parameter after the sample is first produced and after it was shaken, at which point liquid was redistributed throughout the sample. 

\section{Conclusions}

We see that the PFH and nitrogen gas mixture produced a foam whose coarsening rate is such that a foam remains monodisperse over the course of a week. In spite of this, the internal structure of the sample is seen to change dramatically and unexpectedly during this time, progressing from a disordered to more ordered state. The slow rate of this process was surprising as the prevailing opinion was that this was a rapid process, the ordering of the foams occurring directly after crystallisation.

The slow rate of this process allows it to be easily imaged using convenient lab-based X-ray tomography. From this data, the coordination number, translational order parameter and BOOP have all been shown as useful metrics for charting this process. We see that the foam produces regions of fcc and hcp ordering, with a clear preference for fcc crystallisation. 

Now that we have shown that dynamic crystalline processes may be imaged using lab-based tomography, a much broader range of experiments may now be conducted. We will be able to see how border conditions, crystalline defects and other anomalies influence the crystallisation of these foams. In this way, we will fully expand the original work of Bragg into three dimensions, employing his bubble model as dynamic model of 3D crystal structures. 

%We have seen that the average bubble diameter of the sample does not increase by more than 3\% over the seven days of the experiment, while the dispersity of the sample does not increase significantly during this time. We can thus conclude that the additional PFH gas successfully retards coarsening within our foam. We see, however, that the temporal evolution of foam structure is not simply dependent on coarsening dynamics as, when such effects are slowed significantly, the structure is still seen to change dramatically.

%These structural transitions see the foam ordering in time. The coordination number, translational order parameter and BOOP have been all shown to be useful metrics of this process. The packing fraction, which appears so useful in the description of hard-sphere packings, shows no significant variation with structural changes within the sample. We conclude that the compression of the bubbles, pushing the packing fraction above those values associates with hard spheres, renders the packing fraction as an unsuitable descriptor for these foams.  

%The driving mechanism behind this crystallisation process is still unclear. While coarsening dynamics have been previously linked to relaxation dynamics, the significantly reduced coarsening rate present within our foam suggests this is an unlikely source of the structural re-arrangements reported here. In addition, the increased ordering rate of the foam following physical disturbances of the sample is incongruent with such coarsening arguments. However, such behaviour is in keeping with the dynamics one would expect from the re-distribution of liquid through the sample and the resulting re-arrangements which occur during the drainage process.  

%We have seen that the ordered nature of these foams makes them an ideal system for the study of crystalline structures. It was previously believed that these foam structures ordered quickly, spontaneously ordering on production \cite{Bragg47,vanderNetEtal07}. We now see that this ordering process is far from instantaneous, requiring considerable time for the interior of the sample to evolve from a disordered to ordered arrangement. The slow rate of this process allows it to be easily imaged using modern X-ray tomography. 

%In future we aim to use to use this slow ordering process to investigate many interesting aspects of the crystallisation process. We will be able to see how border conditions, crystalline defects and other anomalies influence the crystallisation of these foams. In this way, we will fully expand the original work of Bragg into three dimensions, employing his bubble model as dynamic model of 3D crystal structures. 

 
\section{Acknowledgements}
%This publication has emanated from research conducted with the financial
%support of Science Foundation Ireland (08/RFP/MTR1083).
%Research also supported by the European Space Agency (MAP
%AO-99-108:C14914/02/NL/SH and AO-99-075:C14308/00/NL/SH) and European Union MPNS COST
%Action MP1106. Thanks to Jason Jensen and the group of Prof. Martin Hegner for the assistance in producing the various 3D printed components used in this work.
%Special thanks to Joe Reville, for many interesting intellectual discussions.

This publication has emanated from research conducted with the financial support of Science Foundation Ireland (08/RFP/MTR1083). Research also supported by the European Space Agency (MAP AO-99-108:C14914/02/NL/SH and AO-99-075:C14308/00/NL/SH) and European Union MPNS COST Action MP1106. Thanks to Jason Jensen and the group of Prof. Martin Hegner for the assistance in producing the various 3D printed components used in this work. Special thanks to Joe Reville, for many interesting intellectual discussions. DW thanks the University of Western Australia for a Gledden Fellowship, during the tenure of which this was completed.



\bibliography{Thesis.bib}
\bibliographystyle{unsrt}

\end{document}








